\subsection{Future Directions}
This second result tells us that entanglement is a crucial resource necessary to achieve good bounds in compression, and exhibits the existence of ensembles incompressible in the absence of entanglement.
Even so, Anshu, Leung, and Touchette exhibit ensembles which are incompressible even in the presence of arbitrary entanglement.
Their result, however, is fundamentally classical in construction.

This begs the question of whether there may be a measure of ``quantumness'' which agrees with the above two incompressibility results.
In particular, the second result shows existence of a family of ensembles for which the compression rate can approach 0 in the presence of entanglement,
and the first gives explicit examples of classical constructions for which entanglement does not aid in compression rates at all.
Though Hadiashar and Nayak only show existence of such a family,
it would not be surprising if the ensembles were highly quantum in some sense.
This may explain why entanglement is much more useful in the second case than the first.

If there were then such a metric of ``quantumness'' by how much entanglement aids in compression,
it would additionally be of interest to see how this metric would agree with other well-studied notions such as separability.