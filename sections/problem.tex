\subsection{Problem Description}
Formally, quantum compression can be stated as a game between two players.
Ashwin and Bert first agree upon an \emph{ensemble} 
\(E \coloneq \{(p_x, \rho_x)\}_x\)
ranging over some alphabet \(x \in X \).
We ask for \(p_x\) to sum to 1
so that \(x \mapsto p_x \) is a valid probability mass function over \(X \).
According to this above probability, a referee samples \(x_1, x_2, \dots, x_n\)
and prepares \(\rho_{x_1}^{\tsf{C}_1} \ot \rho_{x_2}^{\tsf{C}_2} \ot \cdots \ot \rho_{x_n}^{\tsf{C}_n} \)
say on some register \(\tsf{C} = \bigotimes_{k=1}^n \tsf{C}_k\) for Ashwin. 
Ashwin is tasked with sending this above state to Bert using as few resources as possible.
In blind compression, Ashwin is not given a (classical) copy of the sampled \(x_i\).
In contrast, for visible compression Ashwin would have access to such a copy of the samples.
Additionally, Ashwin and Bert may share some entanglement \(\ket{\phi}^{\tsf{AB}}\).
If they do, we say that the protocol is entanglement-assisted.
Furthermore, in the special case where \(n = 1\), the protocol is said to be one-shot.

In practice, we associate to the ensemble the classical-quantum state 
\[\rho^\tsf{XC} \coloneq \sum_{x \in \Sigma} p_x \opproj{x}^\tsf{X} \ot \rho_x^\tsf{C}\]
and take a block of \(n\) samples to be represented by
\[\left(\rho^{\tsf{XC}}\right)^{\ot n} = 
    \sum_{x_1x_2\dots x_n}p_{x_1}\dots p_{x_n}
    \opproj{x_1}^{\tsf{X}_1} \ot \dots \ot \opproj{x_n}^{\tsf{X}_n}
    \ot \rho_{x_1}^{\tsf{C}_1} \ot \dots \ot \rho_{x_n}^{\tsf{X}_n}
\]
The compression then comprises of a pair of encoding and decoding channels
\[ \mce : \tsf{C}_1\tsf{C}_2\dots\tsf{C}_n\tsf{A} \to \tsf{TA'},
    \qquad \mc{D} : \tsf{TB} \to \tsf{C}_1\tsf{C}_2\dots\tsf{C}_n\tsf{B'}\]
for which the communication cost of this protocol is \(\log \dim \tsf{C'}\)
and the entanglement cost (as in~\cite{BabHadiashar2020entanglementcostof})
is \(\log \dim \tsf{A} = \log \dim \tsf{B} \)
representing the number of ebits required.

We also consider two different characterizations of error as in~\cite{winter2002compression}.
For some error parameter \(\eps\),
in the local regime we ask that for all \(i\),
\[\Delta\left(
        \Tr_{\tsf{A'B'C}_1\dots\tsf{C}_{i-1}\tsf{C}_{i+1}\dots\tsf{C}_n}
            \circ \mc{D}\circ\mce
            \left(\rho_{x_1}^{\tsf{C}_1}\ot \dots \ot \rho_{x_n}^{\tsf{C}_n}
                \ot \opproj{\phi}^{\tsf{AB}}\right),
        \rho_{x_i}
    \right) \le \eps \]
where \(\Delta(\rho, \sigma) \coloneq \norm{\rho - \sigma}_1\) is the trace distance between
\(\rho\) and \(\sigma\).
Essentially, the trace distance must be bounded for each sample in the block.

Alternatively, the global regime asks for the global error to be bounded:
\[\Delta\left(
    \Tr_{A'B'} \circ \mc{D} \circ \mce \left(
        \rho_{x_1}^{\tsf{C}_1}\ot \dots \ot \rho_{x_n}^{\tsf{C}_n}
            \ot \opproj{\phi}^{\tsf{AB}}        
    \right),
    \rho_{x_1}\ot \dots \ot \rho_{x_n}
\right) \le \eps\]
Note that the global error criterion implies the local.